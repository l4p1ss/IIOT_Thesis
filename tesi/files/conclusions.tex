\chapter{Conclusioni}
Dalla visione generale di WSN e SDN, alla loro unione nelle SD-WSN, e in particolare mediante l'architettura SDN-WISE, la comunit� scientifica si aspetta un grande giovamento per il mondo IoT e le azienda stanno investendo molto per uniformarsi agli standard previsti per l'Industria 4.0. Per quanto sperimentato e ottenuto in questa tesi si pu� concludere che la strada � promettente, con l'unico vincolo che il framework SDN-WISE consente di simulare una WSN, ma non verificare nella realt� i risultati ottenuti. Dunque � doveroso, alla luce dei promettenti risultati raccolti, passare a dei veri sensori.
\newline
Da questo progetto di tesi possiamo proporci la domanda, ma SDN ha davvero senso nella sua applicazione industriale? La risposta � ovviamente s�, dato che i vantaggi sono molteplici, ovvero:
\newline
- \textit{Affidabilit�}
\newline
- \textit{Produttivit�}
\newline
- \textit{Sicurezza}
\newline
- \textit{Riduzione dei costi}
\newline
Ovviamente ci sono anche degli aspetti peggiorativi, come ad esempio la non predicibilit� dei vari risultati ottenuti.
\newline
Una possibile soluzione futura � l'introduzione della tecnologia 5G, che in realt� tanto futura non � visto che si parla gi� da molto tempo di questa nuova e rivoluzionaria tecnologia.
\newline
A livello di tempi di risposta si guadagnerebbe in latenza, dato che la latenza per la tecnologia 5G sappiamo essere di 1ms, dato che ha un data-rate molto pi� alto rispetto alle tecnologie attuali. Da ci� possiamo dedurre che quindi il ritardo lettura sensore ed il ritardo di trasmissione riceverebbero una notevole riduzione, aumentando ancora di pi� la produttivit� e diminuendo cos� i ritardi, anche nei casi pi� sfavorevoli.
\newline
Un altro aspetto da considerare della tecnologia 5G � che il Controller SDN ha a disposizione uno scheduler che schedula i vari flussi, alleggerendo di non poco il lavoro al controller.
\newline
\newline
Un possibile sviluppo futuro di questa tesi potrebbe essere, ad esempio, l'introduzione della tecnologia 5G e l'implementazione di pi� controller, per andare poi ad analizzare i dati ottenuti e compararli con la soluzione attuale. 
\newline
\newline
\subsubsection{Ringraziamenti}
Si ringraziano il Professor Francesco Chiti ed il Dottor Michele Bonanni per l'aiuto e la collaborazione forniti durante questi mesi per lo sviluppo di questo progetto di tesi. Nonostante le limitazioni che hanno contraddistinto questo periodo, non hanno fatto mai mancare il loro supporto e la loro volont� di portare a compimento questo progetto innovativo che ha presentato molte sfide durante tutto lo svolgimento. 