\chapter{Internet of Things}\label{ch:chapter2}

L'Internet of Things (IOT) � l'ultima tecnologia sviluppata nella lunga e continua rivoluzione nel mondo della comunicazione ed elaborazione dati. 
\newline
La sua dimensione ed influenza nella  vita di tutti i giorni, nel mondo del business e nel mondo politico ha schiacciato ogni tipo di tecnologia avanzata sviluppata in precedenza.

\section{Things of the Internet of Things}
Internet of Things (IOT) � un termine che si riferisce ai collegamenti in sviluppo che spaziano dai piccoli dispositivi smart fino agli applicativi con piccoli sensori. Il tema dominante � la possibilit� di mettere in comunicazione persone con oggetti ed oggetti con oggetti stessi. Attualmente l'internet che conosciamo oggi permette di interconnettere bilioni di industrie e di oggetti personali che usualmente si interfacciamo con sistemi Cloud. Invece l'IOT � principalmente guidato da dispositivi embedded. Questi dispositivi sono usualmente a bassa larghezza di banda, bassa ripetizione per l'acquisizione dati ed a bassa larghezza di banda per utilizzo dati che comunicano fra loro attraverso la user interface. 

\section{Evoluzione}
Le evoluzioni di Internet che hanno portato allo sviluppo dell'IOT sono quattro:
\newline
- INFORMATION TECHNOLOGY: PCs, servers, routers, firewalls e pi� in generale dispositivi IT con connessione internet.
\newline
- OPERATIONAL TECHNOLOGY (OT): macchine a applicativi non costruiti da compagnie IT, come ad esempio macchinari medici, SCADA(supervisory control and data acquisition) e processi di controllo.
\newline
- PERSONAL TECHNOLOGY: smartphones, tablets ed ebook reader, acquistati da vari clienti, che utilizzano connessione ad internet. Inoltre i dispositivi presentano varie forme di connettivit� wireless.
\newline
- SENSOR/ACTUATOR TECHNOLOGY: singoli dispositivi che utilizzano connettivit� wireless che fanno parte di grandi sistemi.

\section{Strati dell'Internet of Things}
La letteratura tecnica e quella business si focalizzano su due aspetti principali dell'IOT ovvero gli oggetti che sono collegati ad Internet che interconnette loro stessi.
\newline
La migliore visione dell'IOT � vederlo come un enorme sistema che consiste in cinque strati:
\newline
- Sensori e attuatori: questi sono oggetti. I sensori osservano l'ambiente e riportano delle misurazioni. Gli attuatori operano direttamente su quell'ambiente.
\newline
- Connectivity: un dispositivo pu� collegarsi via wireless o via cavo ad un network e mandare collezioni di dati al data center ( sensore) oppure ricevere comandi da un controllore (attuatori).
\newline
- Capacity: il network che supporta i dati deve essere capace di supportare un enorme flusso di dati.
\newline
- Storage: necessita di un largo storage per salvare e per mantenere i dati (backup). Generalmente avviene lato Cloud.
\newline
- Data Analytics: per un gran numero di dispositivi viene generato un flusso dati enorme che richiede di essere analizzato e processato.
